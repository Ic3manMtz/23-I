\section*{Ejercicio 2}
\noindent La ciencia de datos va más allá de las técnicas y algoritmos, requiriendo habilidades multidisciplinarias para traducir entre la tecnología y las preocupaciones comerciales. Por lo que hay un proceso para poder lograrlo, a continuación veremos estos pasos:

\subsection*{$\Rightarrow$ Encuadre del problema.}
\noindent Lo más importante es definir de manera precisa un problema antes de resolverlo. Como científico de datos, se menciona que es común recibir aportaciones ambiguas, por lo que es necesario trabajar con los solicitantes y formular las preguntas adecuadas para comprender el problema en profundidad.\\
Se presentan una serie de preguntas clave que se deben plantear al inicio del proceso, con el objetivo de identificar a los clientes, comprender el proceso de venta actual, recopilar información sobre los clientes potenciales y conocer los niveles de servicio existentes.
Además se enfatiza la importancia de adentrarse en la perspectiva del cliente, en este caso el vicepresidente de ventas, para comprender su visión del problema. Este conocimiento será invaluable para el análisis de datos y la presentación de puntos de vista.\\
Finalmente, es necesario realizar preguntas más precisas una vez que se tenga un conocimiento razonable del dominio, con el objetivo de comprender exactamente qué problema desea resolver el cliente. Se sugieren dos preguntas específicas relacionadas con la optimización del embudo de ventas y las áreas que el vicepresidente considera no optimizadas en el presente.

\subsection*{$\Rightarrow$ Recopilar los datos brutos}
\noindent En este paso se resalta la importancia de considerar los datos necesarios en un proyecto de ciencia de datos. Se plantean varias preguntas clave relacionadas con los datos de la base de datos CRM, como cuáles son necesarios, cómo extraerlos y en qué formato deben almacenarse para facilitar su análisis.\\
Es importante ser un científico de datos ético, con un enfoque tanto en la seguridad como en la privacidad de los datos. Además de tener cuidado de no extraer información de identificación personal de la base de datos y asegurar que toda la información en el archivo sea anónima y no se pueda rastrear hasta clientes específicos. La mayoría de los proyectos de ciencia de datos se utilizan datos existentes que se están recopilando en tiempo real. Sin embargo, ocasionalmente puede ser necesario liderar esfuerzos para recopilar nuevos datos, lo cual implica trabajo de ingeniería y tiempo para obtener resultados satisfactorios.

\subsection*{$\Rightarrow$ Procesar y explorar los datos}
\noindent Se revisan los datos extraídos para comprender el significado de cada columna y se verifica la presencia de valores faltantes, intervalos adecuados y el formato correcto. Se decide qué hacer con los valores perdidos o corruptos, ya sea eliminarlos o utilizar valores predeterminados.\\
Una vez completada la limpieza de datos, se inicia la exploración para obtener conclusiones. Se busca descubrir qué información contienen los datos y qué partes son relevantes para responder a las preguntas planteadas. Se pueden examinar patrones en los datos dividiéndolos en grupos según si los clientes se han convertido o no.

\subsection*{$\Rightarrow$ Realizar análisis}
\noindent Después de recopilar suficiente información, es momento de crear un modelo que responda a la pregunta inicial. Para ello, se utilizan técnicas de Aprendizaje Automático (Machine Learning) que se basan en conjuntos de datos y vectores de características. Para desarrollar el modelo, se elige la Regresión Logística, un algoritmo simple de clasificación dentro del Aprendizaje Automático Supervisado. Este algoritmo aprende a partir de ejemplos etiquetados y proporciona una predicción binaria y una probabilidad de conversión.\\
Después de aplicar la Regresión Logística y ajustar los parámetros, se obtienen resultados emocionantes con una precisión del 95\%. Sin embargo, es importante comunicar estos resultados de manera convincente y comprensible para el cliente.

\subsection*{$\Rightarrow$ Comunicar los resultados}
\noindent En este paso, se aborda la importancia de comunicar los resultados del modelo de Machine Learning de manera efectiva al cliente. La habilidad de comunicación es fundamental en el campo de la ciencia de datos, ya que implica traducir el trabajo realizado en información comprensible y práctica. Es decir, es la necesidad de contar una historia basada en los resultados obtenidos del análisis exploratorio y del modelo predictivo. La historia debe incluir conclusiones relevantes que respondan a las preguntas más importantes para el cliente.\\
En la siguiente etapa, se reúnen con el cliente y se explican las conclusiones. El cliente se muestra entusiasmado con los resultados y plantea la pregunta sobre cómo pueden utilizar mejor estos hallazgos. Como científico de datos, se espera que no solo se analicen los datos, sino que también se brinden recomendaciones sobre cómo utilizar los resultados.